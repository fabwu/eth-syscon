\documentclass[a4paper,12pt]{article}
\usepackage[utf8x]{inputenc}
\usepackage[T1]{fontenc}
%\usepackage{serohyph}
\usepackage[ngerman]{babel}
\usepackage{latexsym}
\usepackage[pdfpagelabels,plainpages=false]{hyperref} %Für Übersichts-Lesezeichen im PDF
			% Fuer Print besser ’false’
\usepackage{setspace} % Zeilenabstand ändern mi
%\usepackage{sectsty} % section style package
\usepackage[%small
%,compact
]{titlesec}  % Paket um Überschriften einfach zu manipulieren
\usepackage{enumitem} % Paket für Auflistungen zu manipulieren (\setlist{noitemsep})
\usepackage{color} % für Farben

%\titleformat{ command }[ shape ]{ format }{ label }{ sep }{ before-code }[ after-code ]
\titleformat{\section} % titlesec
  {\normalfont\bfseries} %
  {\bfseries \thesection}
  {.5em}
  {}
\titlespacing{\section} % Um angehenden Textblock nach \section layouten.
             {0pc}{2ex}{3ex}[0pc] % Hiermit wird der Abstand zwischen Titel und Text verkelinert (=0)
%{0pc}{2ex}{0ex}[0pc] Abstand zum linken Rand |Abstand zum oberen Text | Abstand zum unteren Text | Abstand zum rechten Rand
%ACHTUNG letztes Klammerpaar eckige Klammern!!
%
\hypersetup{		% von package hyperref
colorlinks=true,        % Links farbig ('true') oder nicht ('false'). Fuer
                        % Print besser 'false'
linkcolor=red,          % Farbe fuer die internen Links
urlcolor=blue,          % Farbe fuer externe Links (http://...)
pdfborder={0 0 0},      % wenn colorlinks nicht gesetzt ist gibt es einen
                        % Rand dieser Farbe (R, G, B} um den Link
                        % geoeffnet werden
pdfpagemode=UseOutlines,% andere Moeglichkeiten: 'None', 'UseThumbs'
                        % und 'FullScreen'
pdftitle={Titel des Dokuments}, pdfauthor={Autor(en)},
pdfsubject={Thema}, pdfkeywords={Keywords} }

\addtolength{\textwidth}{2cm} %Textblcok-Breite
\addtolength{\oddsidemargin}{-1cm} %Abst. 1 inch von linkem Rand
\addtolength{\headsep}{-2cm} %Abst zu Kopfzeile
\addtolength{\textheight}{3cm} % Textblock-Länge
%
\def\headtitle{System Construction} % ToDo: Wiedergebbare Titel für den gesammten Text \headtitle
%
\title{\headtitle}
\author{VIS}
\makeindex

\begin{document}
\pagestyle{empty}
\parindent 0pt	% kein Erstzeileneinzug für das ganze Dokument
\flushleft % linksbündig
\setlist{noitemsep} %Kein Zeilenabstand zwischen Listen
%
%\maketitle
%
%\color{blue}
\color[rgb]{0,0,.5}
{\LARGE \headtitle\par}
\hrulefill\\
%
\section*{Oral Exam Report}
\vspace{.5cm}
%\begin{onehalfspace} %geht nur bei tabular, nicht bei "`tabbing"'
  \begin{tabbing}
    \quad \= Examiner:XXXX \= XXXXXXXXXXX \kill % Abstandgebende-Zeile
    \>Course:		\>	\headtitle\\[1ex]
    \>Examiner: 	\>	F. Friedrich\\[1ex]	%ToDo
    \>Protocol: 	\>	P. Reed\\[1ex]	%ToDo
    \>Semester:	\>	HS19\\[1ex]	%ToDo
    \>Datum: 		\>	2020-02-03%ToDo
  \end{tabbing}
%\end{onehalfspace}%
%
%\vspace{0.1cm}
\hrulefill\\
\color{black}
\vspace{.7cm}

%ToDo
\section{Explain how SPI work (draw waves)}
CLK, CS/SS, MOSI, MISO \\
Pull CS low, start toggling clock, set/reset MOSI/MISO, sample on clock edge, pull CS high

\section{How to wire master and slaves together?}
CLK/MOSI/MISO same for all slaves and separate CS for each slave \\
can also use daisy-chain

\section{How does CS/SS work?}
If SS is not pulled low then data wires are in tri-state (don't know if that's correct)

\section{Assume you have RPI with Minos how to protect against infinite loops?}

Watchdog, reset timer in background task

\section{What do we test if we reset timer in periodic task?}

We test if interrupts (scheduling) still work

\section{Explain the programming model of A2}

Active Objects, AWAIT, EXCLUSIVE, ACTIVE

\section{What are the states of a active object in A2?}

Ready, Running, Terminated, Condition, Lock (draw state diagram)

\section{How to implement conditions or locks?}

Queues in object header which are protected by fine grained locks implemented in hardware 
\bigbreak
Then they run out of questions and asked me for what the CMPXCHG instruction in x86 can be used. I thought they mean CAS and explained how to implement a spin-lock with it but CMPXCHG is actually used to implement CAS.

\end{document}
