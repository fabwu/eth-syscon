\documentclass[10pt, a4]{article}
\usepackage{a4wide}
\usepackage[latin1]{inputenc}
\usepackage{graphicx}
\usepackage{color}
\usepackage{amssymb,amsmath}
\usepackage{listings}
\usepackage{helvet}
\usepackage{fancyvrb}
\renewcommand*\familydefault{\sfdefault}

\usepackage{color}
\usepackage[hidelinks,colorlinks=true,linkcolor=blue,urlcolor=blue]{hyperref}
% listings: http://mirror.switch.ch/ftp/mirror/tex/help/Catalogue/entries/listings.html
\usepackage{listings}
\usepackage{xspace}
% longtable: http://mirror.switch.ch/ftp/mirror/tex/help/Catalogue/entries/longtable.html
\usepackage{longtable}
\usepackage{array}

\usepackage[contents={},opacity=1,scale=1.5,color=black!60,all]{background}
\usepackage{graphicx}
\usepackage{lastpage}
%\usepackage[colorlinks=false, pdfborder={0 0 0}]{hyperref}
\parskip 5pt
\parindent 0cm

\usetikzlibrary{calc}

\definecolor{lightgrey}{rgb}{0.5,0.5,0.5}
\definecolor{darkgrey}{rgb}{0.4,0.4,0.4}

\newcommand{\changefont}[3]{\fontfamily{#1}\fontseries{#2}\fontshape{#3}\selectfont}

\newcommand{\progfont}{\changefont{pcr}{m}{n}}
\newcommand{\kwfont}{\changefont{pcr}{b}{n}}

\lstdefinelanguage{raw}{}
\lstdefinelanguage{Oberon}[]{Oberon-2}%
  {morekeywords={OBJECT,SELF,%
   HUGEINT,% Basic Types
   AWAIT},% Built in functions
   %sensitive=f,%
   %alsoother={},% list here in lower case if keyword some where else wrongly highlighted
    morecomment=[s][\color{red}]{(*!}{!*)}
  }[keywords]%
\lstset{language=Oberon,
basicstyle=\small\progfont,keywordstyle=\kwfont ,identifierstyle=\progfont,
commentstyle=\color{darkgrey}, stringstyle=, showstringspaces=false, %keepspaces=true,
numbers=none, numberstyle=\tiny, stepnumber=1, numbersep=5pt, captionpos=b,
columns=flexible, % flexible, fixed, fullflexible
framerule=0.1mm,frame=shadowbox, rulesepcolor=\color{blue}, % frame = shadowbox
xleftmargin=2mm,xrightmargin=2mm,
breaklines=true,                % break long lines
breakatwhitespace=true,         % break lines only at white space
}
\renewcommand{\lstlistingname}{Fig.}


\newcommand{\mylogo}{
    \begin{tikzpicture}[overlay]
    \node [below right, inner sep=5mm] (logol) at (current page.north west)   {\includegraphics[width=5cm]{../img/ETHlogo.pdf}};
%    \node [below left, inner sep=5mm] (logo2) at (current page.north east)   {\includegraphics[width=6.5cm]{../img/NCTU.png}};
    %\node [below left, inner sep=5mm] (logor) at (current page.north east)   {\parbox{6cm}{Department of Computer Science \\ Lecturer Felix Friedrich}};
    %\draw [color=black, thin] (logol.east) -- (logor.west);
    \end{tikzpicture}
    }
\SetBgContents{\mylogo}
\SetBgPosition{current page.north west}% Select location
\SetBgOpacity{1.0}% Select opacity
\SetBgAngle{0.0}% Select roation of logo
\SetBgScale{1.0}% Select scale factor of logo
\SetBgOpacity{1.0}% Select opacity

\usetikzlibrary{shapes,shadows}
  \newcommand{\whitebox}[2]{
    \begin{center}
      \begin{tikzpicture}
        \node [abstractbox, fill=white] (box) {\begin{minipage}{0.95\linewidth} #2 \end{minipage}};
        \node[abstracttitle, right=10pt] at (box.north west) {\footnotesize #1};
      \end{tikzpicture}
    \end{center}
  }
\tikzstyle{abstractbox} = [draw=black, fill=white, rectangle,
inner sep=10pt, style=rounded corners, drop shadow={fill=black,
opacity=1}]
\tikzstyle{abstracttitle} =[fill=white]

\begin{document}
\DefineShortVerb{\@}
\newcommand{\assnr}{4}
\newcommand{\assdate}{2019}

\newcommand{\AZ}{\ensuremath{\mathcal{A}_{2}}\xspace}

\pagestyle{myheadings}\markright{System Construction Course \assdate, \assnr}
\thispagestyle{empty}
\phantom{xx}\vspace{-2.7cm}
%\rule{\textwidth}{0.3mm}
\begin{center}
\begin{minipage}{0.95\textwidth}
\begin{center}
System Construction Course 2019,
\\[1em] \textbf{Assignment \assnr} \\[1em]
Felix Friedrich, ETH Z\"{u}rich
\end{center}
\end{minipage}
\rule{\textwidth}{0.3mm}
\end{center}


\subsection*{Introduction}
Minos supports preemptive scheduling under certain preconditions that were discussed in the lectures. Background tasks execute in a round robin fashion in the background. Periodic tasks can preempt background tasks. The command scheduler of Minos is an example of a background task. We can protect ourselves from commands executing an infinite loop with a watchdog.

\whitebox{Lessons to Learn}{
\begin{itemize}
\item Learn to know the task scheduling mechanism of Minos.
\item Understand and apply the mechanism of a watchdog.
\end{itemize}
}
\subsection*{Preparation}
\begin{enumerate}
\item Update your repository
\item Open a console in directory \href{https://svn.inf.ethz.ch/svn/lecturers/vorlesungen/trunk/syscon/2019/shared/assignments/assignment4}{assignments/assignment4}
\item Make sure you have a proper kernel running on your RPI. If you need to recompile the kernel, you can do so by calling @oberon execute MakeMinos.txt@. You can then use module loading and are not required to link the kernel again for the rest of this exercise.
\end{enumerate}


\section{Watchdog and Tasks}
Module
\href{https://svn.inf.ethz.ch/svn/lecturers/vorlesungen/trunk/syscon/2019/shared/assignments/assignment4/Minos/RPI.Kernel.Mod}{Minos/RPI.Kernel.Mod}
contains procedures to setup the (largely undocumented) ARM watchdog registers @WDOG@ (Watchdog) and @RSTC@ (Reset Configuration). Bits 0 to 19 of the @WDOG@ register provide a countdown register that, once it hits 0 will make the system reboot, provided register @RSTC@ is set up accordingly. The frequency of the counter is $2^{16}$ Hz, thus a maximum of 16 seconds can be set for the watchdog countdown. Please refer to the implementation of @Kernel.StartWatchdog@ in order to understand the semantics.

Once the countdown is activated, software must periodically update the @WDOG@ register in order to prevent a reboot. In the case of a failed program, the watchdog is no longer updated which should result in a reboot of the system.
\begin{enumerate}
\item Complement procedure @Tasks.InstallWatchdog@ to enable the watchdog and to install a {\em background} task that (periodically) resets the watchdog.
\item Test the watchdog by executing an infinite loop as a command.
\item What happens when you install the watchdog resetter as a {\em periodic} task? What does it actually test?
\end{enumerate}
The starting point for this exercise is provided as module \href{https://svn.inf.ethz.ch/svn/lecturers/vorlesungen/trunk/syscon/2019/shared/assignments/assignment4/Tasks.Mod}{Tasks.Mod}. It contains some example commands for installing tasks.

\subsection*{Documents}
\begin{itemize}
\item System Construction Lecture 4 slides from the course-homepage \\ \url{http://lec.inf.ethz.ch/syscon}
%\item \href{https://svn.inf.ethz.ch/svn/lecturers/vorlesungen/trunk/syscon/2015/shared/documents/Minos/Minos Technical Documentation.pdf}{Minos Technical Documentation} in @documents/Minos@ folder of the repository.
\end{itemize}

%\begin{itemize}
%\item ARM Architecture Reference Manual, ARMv7-A and ARMv7-R edition: \\ @DDI0406C_C_arm_architecture_v7_reference_manual.pdf@ in folder \texttt{documents/rpi}
%\item Cortex-A7 MPCore Technical Reference Manual: \\ @DDI0464F_cortex_a7_mpcore_r0p5_trm.pdf@ in folder \texttt{documents/rpi}
%\item BCM2836 Technical Addendum: \\ @QA7_rev3.4.pdf@ in folder \texttt{documents/rpi}
%\end{itemize}


\end{document}
